In September 2024 the Internationale Stiftung Mozarteum Salzburg, together with musicologist Neal Zaslaw, presented the ninth edition of the catalog of Mozart's works, originally published by Ludwig von Köchel in 1862. In this edition the catalog goes up to number 721, bypassing the fateful «626» for the first time and thus abandoning the chronological criterion sought in previous editions. For the most part, these new «Koechel numbers» are pieces that were previously listed within the catalog's many appendices, but some are works recently attributed to the composer. 

This edition presents one of them, a «Serenade, ex C» for string trio (two violins, \emph{primo} and \emph{secondo}, plus a bass ---labeled as \emph{violoncello} in the \emph{particella} and as \emph{basso} on the title page) to which the number K. 648 has been assigned. This is a rather unusual work, presumably dating from the late 1760s. A copy of the \emph{particelle} is included in the collection of the German composer Carl Ferdinand Becker (1804-1877) and was digitized by the Leipziger Städtischen Bibliotheken in 2018. In September 2024, in conjunction with the publication of the new catalog edition, the piece was also performed again, presumably for the first time since it was written. 

This edition of the work is naturally based on this single copy. It is a critical--practical edition in the sense that, although all readings other than the manuscript are indicated graphically (using square brackets and dashed lines for the slurs), the aim is to ensure that these do not distract performers when performing the piece. We do not include an appendix detailing the readings adopted: the curious reader can simply consult the manuscript (available at \href{http://digital.slub-dresden.de/id454516029}{http://digital.slub-dresden.de/id454516029}) and draw his or her own conclusions. We also sought to respect some idiosyncrasies of the writing: the occasional «voice» writing in the string parts and the lack of markings of tuplets. Others, on the other hand, were not taken into account: the «kneed beams» (for instance, we replace 
    \begin{lilypond}[insert=inline,voffset=-1.5em]
        \relative c'{ \omit Score.Clef  \omit Score.TimeSignature \stemUp \tweak Beam.positions #'(-1 . -1) c8 \stemDown c' \stemUp c, \stemDown c'}
    \end{lilypond} 
    with 
    \begin{lilypond}[insert=inline, voffset=-1em]
        \relative c'{ \omit Score.Clef  \omit Score.TimeSignature c8 c'  c,  c'}
    \end{lilypond}
), the number of bars per system, and some stem directions. We believe that some of these particularities of the notation possess illocutionary force and convey intentions to performers. 

This edition is presented as an open edition: it was made entirely using Lilypond and Lua\LaTeX\ and its source code is available in an open repository on GitHub (\href{https://github.com/fsarud/mozart648}{https://github.com/fsarud/mozart648}). Any corrections or changes can be made using that source code and recompiled to obtain the print version; also, if desired, those changes can be suggested for inclusion in the edition via a pull request. The music typography work was done by Luca Mariano and Federico Sarudiansky (fsarud at gmail dot com), who was also in charge of the final edition. 

Buenos Aires, September 2024.
